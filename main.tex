\documentclass{beamer}
\usepackage[utf8]{inputenc}

\usetheme{Madrid}
\usecolortheme{default}

%------------------------------------------------------------
%This block of code defines the information to appear in the
%Title page
\title[Monolithic vs. Microservices] %optional
{Monolithic vs. Microservice Architecture: A Performance and Scalability Evaluation}

\subtitle{Análise de Desempenho e Escalabilidade}

\author[Jefferson Uchoa Ponte] % (optional)
{Jefferson Uchoa Ponte\\
Baseado no artigo de: Jose Luis Vazquez-Poletti, Adrian Mallmann, Guilherme Zambon, Marco Netto}

\institute[UECE] % (optional)
{
  Universidade Estadual do Ceará (UECE) \\
  Engenharia de Software \\
  Prof. Matheus Paixão
}

\date[Junho 2024] % (optional)
{Junho de 2024}

%End of title page configuration block
%------------------------------------------------------------

\begin{document}

%The next statement creates the title page.
\frame{\titlepage}

\section{Introdução}

\begin{frame}
\frametitle{Introdução}
\begin{itemize}
    \item Microservices: Melhora disponibilidade, tolerância a falhas, escalabilidade e agilidade.
    \item Monolithic: Aplicações tradicionais com uma única base de código.
\end{itemize}
\end{frame}

\section{Objetivo do Estudo}

\begin{frame}
\frametitle{Objetivo do Estudo}
Comparar a performance e escalabilidade das arquiteturas monolítica e de microserviços:
\begin{itemize}
    \item Aplicação web de referência.
    \item Tecnologias: Java vs. C\# .NET.
    \item Ambientes: Local, Azure Spring Cloud, Azure App Service.
\end{itemize}
\end{frame}

\section{Metodologia}

\begin{frame}
\frametitle{Metodologia}
\begin{itemize}
    \item Implementação de quatro versões da aplicação:
    \begin{itemize}
        \item Monolítica em Java e C\# .NET.
        \item Microserviços em Java e C\# .NET.
    \end{itemize}
    \item Experimentos realizados em três ambientes de deployment:
    \begin{itemize}
        \item Local
        \item Azure Spring Cloud
        \item Azure App Service
    \end{itemize}
    \item Critérios de avaliação:
    \begin{itemize}
        \item Performance em máquina única
        \item Escalabilidade vertical (aumento de recursos na mesma máquina)
        \item Escalabilidade horizontal (adição de instâncias)
        \item Impacto da tecnologia (Java vs. C\# .NET)
    \end{itemize}
\end{itemize}
\end{frame}

\begin{frame}
\frametitle{Configuração dos Experimentos}
\begin{itemize}
    \item Configuração das máquinas:
    \begin{itemize}
        \item Local: Máquina com processador quad-core e 16GB de RAM.
        \item Azure: Configurações variadas de instâncias (pequenas a grandes).
    \end{itemize}
    \item Ferramentas de monitoramento e teste:
    \begin{itemize}
        \item JMeter para simulação de carga.
        \item Azure Monitor para métricas de desempenho.
    \end{itemize}
    \item Cenários de teste:
    \begin{itemize}
        \item Testes de carga com diferentes níveis de concorrência.
        \item Medição de tempo de resposta, throughput e uso de recursos.
    \end{itemize}
\end{itemize}
\end{frame}

\section{Resultados Principais}

\begin{frame}
\frametitle{Resultados Principais}
\begin{itemize}
    \item Desempenho em Máquina Única: Monolíticas têm melhor desempenho.
    \item Java vs. .NET: Java melhor em máquinas poderosas, .NET melhor em máquinas menos potentes.
    \item Escalabilidade Vertical vs. Horizontal: Vertical é mais econômica no Azure.
    \item Limite de Instâncias: Aumento excessivo degrada a performance.
    \item Impacto da Tecnologia: Implementação (Java ou C\# .NET) não afeta escalabilidade.
\end{itemize}
\end{frame}

\section{Conclusão}

\begin{frame}
\frametitle{Conclusão}
\begin{itemize}
    \item Monolítica: Simplicidade, facilidade de teste, deploy, debug e monitoramento.
    \item Microserviços: Melhor para aplicações complexas e grandes, com desafios em comunicação e gestão de dados.
\end{itemize}
\end{frame}

\section{Discussão}

\begin{frame}
\frametitle{Discussão}
\begin{itemize}
    \item Microserviços: Vantajosos para sistemas com alta demanda e complexidade.
    \item Monolíticas: Adequadas para pequenas empresas ou sistemas de menor escala.
\end{itemize}
\end{frame}

\end{document}
