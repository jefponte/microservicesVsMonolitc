\documentclass{beamer}
\usepackage[utf8]{inputenc}
\usepackage{graphicx} % Para incluir gráficos
\usepackage{caption}

\usetheme{Madrid}
\usecolortheme{default}

%------------------------------------------------------------
%This block of code defines the information to appear in the
%Title page
\title[Monolithic vs. Microservices] %optional
{Monolithic vs. Microservice Architecture: A Performance and Scalability Evaluation}

\subtitle{Análise de Desempenho e Escalabilidade}

\author[Jefferson Uchoa Ponte] % (optional)
{Jefferson Uchoa Ponte\\
Baseado no artigo de: Grzegorz Blinowski, Anna Ojdowska, Adam Przybyłek}

\institute[UECE] % (optional)
{
  Universidade Estadual do Ceará (UECE) \\
  Engenharia de Software \\
  Prof. Matheus Paixão
}

\date[Junho 2024] % (optional)
{Junho de 2024}

%End of title page configuration block
%------------------------------------------------------------

\begin{document}

%The next statement creates the title page.
\frame{\titlepage}

\section{Introdução}

\begin{frame}
\frametitle{Introdução}
\begin{itemize}
    \item \textbf{Pioneiros}: Netflix iniciou a transição de monolítica para microserviços em 2009.
    \item \textbf{Oficialmente cunhado}: Termo oficialmente cunhado em 2011, após um ano anunciado na 33rd Degree Conference in Kraków.
    \item \textbf{Popularização}: Ganhou tração em 2014 com a publicação de Lewis e Fowler e os relatos de sucesso da Netflix.
    \item \textbf{Adesão de Empresas Globais}: Amazon, eBay, Spotify, Uber, Airbnb, LinkedIn, entre outras, adotaram microserviços.
\end{itemize}
\end{frame}

\begin{frame}
\frametitle{Desafios e Benefícios}
\begin{columns}
    \column{0.5\textwidth}
    \textbf{Desafios dos Microserviços}:
    \begin{itemize}
        \item Identificação de limites de microserviços.
        \item Orquestração de serviços complexos.
        \item Manutenção da consistência de dados.
        \item Aumento do consumo de recursos.
    \end{itemize}

    \column{0.5\textwidth}
    \textbf{Benefícios dos Microserviços}:
    \begin{itemize}
        \item Melhor escalabilidade horizontal.
        \item Facilidade de manutenção independente.
        \item Maior tolerância a falhas.
    \end{itemize}
\end{columns}
\end{frame}

\begin{frame}
\frametitle{Motivação para o Estudo}
\begin{itemize}
    \item \textbf{Pequenas Empresas}: Avaliar benefícios reais da migração para microserviços.
    \item \textbf{Escalabilidade Vertical vs. Horizontal}: Efetividade para sistemas menores.
    \item \textbf{Evidências Empíricas}: Preencher a lacuna de conhecimento sobre os benefícios reais.
\end{itemize}
\end{frame}

\section{Objetivo do Estudo}

\begin{frame}
\frametitle{Objetivo do Estudo}
\begin{itemize}
    \item Comparar a performance e escalabilidade das arquiteturas monolítica e de microserviços:
    \begin{itemize}
        \item Aplicação web de referência.
        \item Tecnologias: Java vs. C\# .NET.
        \item Ambientes: Local, Azure Spring Cloud, Azure App Service.
    \end{itemize}
\end{itemize}
\end{frame}

\section{Perguntas da Pesquisa}

\begin{frame}
\frametitle{Perguntas da Pesquisa}
\begin{itemize}
    \item \textbf{(RQ1)} Qual a diferença de desempenho entre aplicações monolíticas e microserviços?
    \item \textbf{(RQ2)} Qual abordagem arquitetural oferece melhor escalabilidade?
    \item \textbf{(RQ3)} Em quais circunstâncias a tecnologia de implementação (Java vs C\# .NET) tem vantagens e desvantagens?
\end{itemize}
\end{frame}

\section{Background}

\begin{frame}
\frametitle{Arquitetura Monolítica}
\begin{itemize}
    \item \textbf{Definição}: Aplicações construídas como uma única unidade.
    \item \textbf{Simplicidade}: Fáceis de testar, implantar, depurar e monitorar.
    \item \textbf{Desvantagens}: Complexidade crescente torna difícil modificar e manter.
    \item \textbf{Desempenho}: Melhor em máquina única devido à comunicação intra-processo.
\end{itemize}
\end{frame}

\begin{frame}
\frametitle{Arquitetura de Microserviços}
\begin{itemize}
    \item \textbf{Definição}: Decompõe o domínio de negócios em pequenos serviços autônomos.
    \item \textbf{Autonomia}: Independência facilita manutenção e desenvolvimento.
    \item \textbf{Escalabilidade}: Escala bem horizontalmente.
    \item \textbf{Desafios}: Orquestração, consistência de dados e gestão de transações.
    \item \textbf{Desempenho}: Pode ter overhead devido à comunicação entre processos (IPC), mas compensa em sistemas de alta demanda.
\end{itemize}
\end{frame}

\section{Metodologia}

\begin{frame}
\frametitle{Metodologia}
\begin{itemize}
    \item Implementação de quatro versões da aplicação:
    \begin{itemize}
        \item Monolítica em Java e C\# .NET.
        \item Microserviços em Java e C\# .NET.
    \end{itemize}
    \item Experimentos em três ambientes:
    \begin{itemize}
        \item Local
        \item Azure Spring Cloud
        \item Azure App Service
    \end{itemize}
    \item Critérios de avaliação:
    \begin{itemize}
        \item Performance em máquina única
        \item Escalabilidade vertical
        \item Escalabilidade horizontal
        \item Impacto da tecnologia (Java vs. C\# .NET)
    \end{itemize}
\end{itemize}
\end{frame}

\begin{frame}
\frametitle{Configuração dos Experimentos}
\begin{itemize}
    \item \textbf{Máquinas}:
    \begin{itemize}
        \item Local: PC com Windows 10, Intel Core i7, 32 GB RAM.
        \item Azure: Instâncias variadas (pequenas a grandes).
    \end{itemize}
    \item \textbf{Ferramentas}:
    \begin{itemize}
        \item JMeter para simulação de carga.
        \item Azure Monitor para métricas de desempenho.
    \end{itemize}
    \item \textbf{Cenários de teste}:
    \begin{itemize}
        \item Testes de carga com diferentes níveis de concorrência.
        \item Medição de tempo de resposta, throughput e uso de recursos.
    \end{itemize}
\end{itemize}
\end{frame}

\section{Resultados Principais}

\begin{frame}
\frametitle{Resultados Principais}
\begin{itemize}
    \item \textbf{Desempenho em Máquina Única}: Monolíticas têm melhor desempenho.
    \item \textbf{Java vs. .NET}: Java melhor em máquinas poderosas, .NET melhor em máquinas menos potentes.
    \item \textbf{Escalabilidade Vertical vs. Horizontal}: Vertical é mais econômica no Azure.
    \item \textbf{Limite de Instâncias}: Aumento excessivo degrada a performance.
    \item \textbf{Impacto da Tecnologia}: Implementação (Java ou C\# .NET) não afeta escalabilidade.
\end{itemize}
\end{frame}

\section{Conclusão}

\begin{frame}
\frametitle{Conclusão}
\begin{itemize}
    \item \textbf{Monolítica}: Simplicidade, facilidade de teste, deploy, debug e monitoramento.
    \item \textbf{Microserviços}: Melhor para aplicações complexas e grandes, com desafios em comunicação e gestão de dados.
\end{itemize}
\end{frame}

\section{Discussão}

\begin{frame}
\frametitle{Discussão}
\begin{itemize}
    \item \textbf{Microserviços}: Vantajosos para sistemas com alta demanda e complexidade.
    \item \textbf{Monolíticas}: Adequadas para pequenas empresas ou sistemas de menor escala.
\end{itemize}
\end{frame}

\end{document}
